
\documentclass[journal]{IEEEtran}

\usepackage[ruled,vlined,linesnumbered]{algorithm2e}
\usepackage{graphicx}
%\usepackage[ref, superscript]{cite}
\usepackage[numbers, super]{natbib}
\usepackage{picinpar}
\usepackage{amsmath}
\usepackage{url}
\usepackage{textcomp}
\usepackage{flushend}
\usepackage{colortbl}
\usepackage{soul}
\usepackage{multirow}
\usepackage{pifont}
\usepackage{color}
\usepackage{alltt}
\usepackage[hidelinks]{hyperref}
\usepackage{enumerate}
\usepackage{siunitx}
\usepackage{epstopdf} 
\usepackage{pbox}
\usepackage{adjustbox}
\usepackage{amssymb}
\usepackage{listings}
\usepackage{booktabs}
\usepackage{mathtools}
\usepackage{subfig}
\usepackage{amsfonts}
\usepackage{epsfig}
\usepackage{cleveref}
\usepackage{times} %usa a fonte times 
\usepackage{lastpage} %para obter o número da última página
\usepackage{float}
\pagenumbering{arabic} %Numeração de página

\bibliographystyle{abbrvnat}

\begin{document}

    \title{Otimização do gasto energético em um galpão de bobinas quentes de aço}
    \nocite{*}
    \author{Everson Elias \& Josoe S. Queiroz \& Valentim Moura, Victor Hugo} 
    
    \maketitle
    	
    \begin{abstract}
        Parágrafo.\\
        Parágrafo.\\
        Parágrafo.\\
        Parágrafo.
    \end{abstract}
    
    \begin{IEEEkeywords}
        SSA, 
    \end{IEEEkeywords}
    
    \section{Introduction}
        \subsection{Produção de aço}
        
        \subsection{Armazens e Guindastes}

        \subsection{Escopo do trabalho}

        \subsection{Estrutura do artigo}

        
    \section{Revisão da literatura}
        
    \section{Descrição do problema}
        \subsection{Notação}
        Vamos adaptar a modelagem de \cite{Weckenborg2025} e que deriva de
        \cite{YUAN2017424}. A fim de obter uma modelagem enxuta que tenha uma
        boa performance computacional. Os espaços disponíveis para armazenamento
        são $q,k\in\ \Psi=\{1,...,|\Psi|\}$. O espaço é alocado em uma de duas 
        camadas $\Psi^1$ ou $\Psi^2$. Os espaços do início da primeira camada
        $\Psi^{P_1}\in\Psi$ e do final $\Psi^{P_{max}}\in\Psi$ são o lugar de 
        entrada e saída de bobinas respectivamente. Todos os pontos de entrada 
        com posições $k,q\in\Phi=\{I,O\}\cap\Psi$.
    
    \section{Descrição do algoritmo}


        
    \section{Resultados e análise}


    
    \section{Conclusão}            

    Neste artigo, buscamos aplicar técnicas da Pesquisa Operacional na otimização de paralelismo na progamação de FPGAs por meio da síntese de alto nível. Ao mesmo tempo, isso significou uma revisão das soluções para os problemas do tipo RCPSP.
    
    \bibliography{referencias}

\end{document}
