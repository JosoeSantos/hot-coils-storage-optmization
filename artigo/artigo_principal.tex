
\documentclass[journal]{IEEEtran}

\usepackage[ruled,vlined,linesnumbered]{algorithm2e}
\usepackage{graphicx}
%\usepackage[ref, superscript]{cite}
\usepackage[numbers, super]{natbib}
\usepackage{picinpar}
\usepackage{amsmath}
\usepackage{url}
\usepackage{textcomp}
\usepackage{flushend}
\usepackage{colortbl}
\usepackage{soul}
\usepackage{multirow}
\usepackage{pifont}
\usepackage{color}
\usepackage{alltt}
\usepackage[hidelinks]{hyperref}
\usepackage{enumerate}
\usepackage{siunitx}
\usepackage{epstopdf} 
\usepackage{pbox}
\usepackage{adjustbox}
\usepackage{amssymb}
\usepackage{listings}
\usepackage{booktabs}
\usepackage{mathtools}
\usepackage{subfig}
\usepackage{amsfonts}
\usepackage{epsfig}
\usepackage{cleveref}
\usepackage{times} %usa a fonte times 
\usepackage{lastpage} %para obter o número da última página
\usepackage{float}
\pagenumbering{arabic} %Numeração de página

\bibliographystyle{abbrvnat}

\begin{document}

    \title{Otimização do gasto energético em um galpão de bobinas quentes de aço}
    \nocite{*}
    \author{Everson Elias \& Josoe S. Queiroz \& Valentim Moura, Victor Hugo} 
    
    \maketitle
    	
    \begin{abstract}
        Parágrafo.\\
        Parágrafo.\\
        Parágrafo.\\
        Parágrafo.
    \end{abstract}
    
    \begin{IEEEkeywords}
        SSA, 
    \end{IEEEkeywords}
    
    \section{Introduction}
        \subsection{Produção de aço}
        
        \subsection{Armazens e Guindastes}

        \subsection{Escopo do trabalho}

        \subsection{Estrutura do artigo}

        
    \section{Revisão da literatura}
        
    \section{Descrição do problema}
        \subsection{Notação}
        Vamos adaptar a modelagem de \cite{Weckenborg2025} e que deriva de
        \cite{YUAN2017424}. A fim de obter uma modelagem enxuta que tenha uma
        boa performance computacional. Os espaços disponíveis para armazenamento
        são $q,k\in\ \Psi=\{1,...,|\Psi|\}$. O espaço é alocado em uma de duas 
        camadas $\Psi^1$ ou $\Psi^2$. Os espaços do início da primeira camada
        $\Psi^{P_1}\in\Psi$ e do final $\Psi^{P_{max}}\in\Psi$ são o lugar de 
        entrada e saída de bobinas respectivamente. Todos os pontos de entrada 
        com posições $k,q\in\Phi=\{I,O\}\cap\Psi$.
    
        As bobinas $A_{in}\subseteq{A}$ podem ser transportadas de $I$ para o 
        armazém no intervalo de tempo contínuo $[\sigma^-, \sigma^+]$ e
        conjunto $A_{out}\subseteq{A}$ pode ser transportado para $O$ no intervalo
        $[\omega^-,\omega^+]$. Nisto também consideramos o tempo de movimentação
        $t^{load}_{kq}$ e a energia gasta $E^{load}_{kq}$ para movimentar a ponte
        rolante carregada para o espaço $(k,q)$. Similarmente, tempo de movimentação
        para uma movimentação descarregada $t^{empty}_{kq}$ e o consumo energético 
        $E^{empty}_{kq}$ também ja são definidos. 

        O tempo também é separado em seções $S$ que servem para limitar somente 
        um movimento da ponte rolate para cada seção $s,\hat{s}\in\{1...|S|\}$.

        As variáveis de decisão $\tau^s\in\mathbb{R}$ simbolizam o tempo de 
        início de cada seção $s$. Uma movimentação no estágio $s$ carregada com 
        a bobina $a$ para o espaço $(k,q)$ é definida por $W^{s}_{qk,a}$ e é uma 
        variável binária onde 1 indica a occorrencia da movimentação e 0 o 
        contrário. De forma parecida uma movimentação descarregada na seção $s$
        para a posição $(k,q)$ é denotada por $V^{s}_{k,q}$. Finalmente a variável 
        $x^{s}_{q,a}$ denota que a bobina $a$ está armazenada na posição $q$ 
        durante o intervalo $s$. 

        De acordo com \cite{YUAN2017424} As posições $k$ e $q$ são modeladas de 
        acordo com um array de três inteiros que delimitam as posições em largura,
        profundidade e altura e elas são definidas de acordo com a disposição de
        espaço do armazáem otimizado. 

        \subsection{Modelagem PLIM}

        O objetivo de otimização é a redução do custo energético de operação da 
        ponte rolante dado o espaço delimitado. Para tal queremos minimizar o 
        custo total de movimentação da ponte rolante carregada e vazia.

        \begin{equation}   
            \begin{align*}
                \min E &= \sum_{k \in \Phi} \sum_{q \in \Phi} \sum_{s \in S} \sum_{a \in A} 
                        \left( E_{kq,a}^{\text{load}} \cdot W_{kq,a}^s \right) \\
                    &\quad + \sum_{k \in \Phi} \sum_{q \in \Phi} \sum_{s \in S} 
                        \left( E_{kq}^{\text{empty}} \cdot V_{kq}^s \right)
            \end{align*}
        \label{eq:objetivo}
        \end{equation}

        %% mais opções de formatação em https://claude.ai/public/artifacts/21eeb3b8-a6b8-4752-9433-9b1826e63fe4

        Sujeito a:

        \begin{equation}
            \tau^{1} = 0
            \label{eq:t0}
        \end{equation}

        A igualdade \ref{eq:t0} implica que o tempo inicial modelado é o instante 0.

        \begin{equation}
            \label{}
        \end{equation}


    \section{Descrição do algoritmo}


        
    \section{Resultados e análise}


    
    \section{Conclusão}            

    Neste artigo, buscamos aplicar técnicas da Pesquisa Operacional na otimização de paralelismo na progamação de FPGAs por meio da síntese de alto nível. Ao mesmo tempo, isso significou uma revisão das soluções para os problemas do tipo RCPSP.
    
    \bibliography{referencias}

\end{document}
