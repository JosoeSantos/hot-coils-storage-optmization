
\documentclass[journal]{IEEEtran}

\usepackage[ruled,vlined,linesnumbered]{algorithm2e}
\usepackage{graphicx}
%\usepackage[ref, superscript]{cite}
\usepackage[numbers, super]{natbib}
\usepackage{picinpar}
\usepackage{amsmath}
\usepackage{url}
\usepackage{textcomp}
\usepackage{flushend}
\usepackage{colortbl}
\usepackage{soul}
\usepackage{multirow}
\usepackage{pifont}
\usepackage{color}
\usepackage{alltt}
\usepackage[hidelinks]{hyperref}
\usepackage{enumerate}
\usepackage{siunitx}
\usepackage{epstopdf} 
\usepackage{pbox}
\usepackage{adjustbox}
\usepackage{amssymb}
\usepackage{listings}
\usepackage{booktabs}
\usepackage{mathtools}
\usepackage{subfig}
\usepackage{amsfonts}
\usepackage{epsfig}
\usepackage{cleveref}
\usepackage{times} %usa a fonte times 
\usepackage{lastpage} %para obter o número da última página
\usepackage{float}
\pagenumbering{arabic} %Numeração de página

\bibliographystyle{abbrvnat}

\begin{document}

    \title{Otimização do gasto energético em um galpão de bobinas quentes de aço}
    \nocite{*}
    \author{Everson Elias \& Josoe S. Queiroz \& Valentim Moura, Victor Hugo} 
    
    \maketitle
    	
    \begin{abstract}
        Parágrafo.\\
        Parágrafo.\\
        Parágrafo.\\
        Parágrafo.
    \end{abstract}
    
    \begin{IEEEkeywords}
        SSA, 
    \end{IEEEkeywords}
    
    \section{Introduction}

    O aço é um dos materiais mais fundamentais e versáteis na sociedade moderna, impulsionando indústrias
    que vão da construção civil à manufatura automotiva e de eletrodomésticos. A produção global do aço 
    é massiva e a eficiência energética em cada etapa da cadeia produtiva é crucial para a 
    sustentabilidade econômica e ambiental.

    Um gargalo de grande impacto reside na energia despendida na movimentação 
    de bobinas de aço em ambientes industriais. Nesses locais, a movimentação 
    de cargas pesadas, como as bobinas, é predominantemente realizada por 
    guindastes aéreos, equipamentos que, embora essenciais, têm alto consumo energético.

        Assim, torna-se relevante um estudo que objetiva a otimização do movimento e da
        armazenagem de bobinas de aço. A armazenagem e a movimentação ineficiente não 
        apenas elevam os custos operacionais, mas também podem impactar a produtividade,
        a segurança dos trabalhadores e a integridade do material. A distribuição de
        bobinas em um espaço, se não devidamente planejada, pode gerar percursos desnecessários
        e tempos ociosos, culminando em desperdício de energia.

        Este trabalho busca investigar e propor soluções para a otimização do movimento 
        de bobinas de aço, com foco na redução do consumo de energia associado à cada operação.
        Para isso, será abordada a modelagem do fluxo de bobinas, com base em
        distribuições distintas e cada vez mais complexas, visando a maximização da eficiência energética.

    \section{Revisão da literatura}
        
    \section{Descrição do problema}
        \subsection{Notação}
        Vamos adaptar a modelagem de \cite{Weckenborg2025} e que deriva de
        \cite{YUAN2017424}. A fim de obter uma modelagem enxuta que tenha uma
        boa performance computacional. Os espaços disponíveis para armazenamento
        são $q,k\in\ \Psi=\{1,...,|\Psi|\}$. O espaço é alocado em uma de duas 
        camadas $\Psi^1$ ou $\Psi^2$. Os espaços do início da primeira camada
        $\Psi^{P_1}\in\Psi$ e do final $\Psi^{P_{max}}\in\Psi$ são o lugar de 
        entrada e saída de bobinas respectivamente. Todos os pontos de entrada 
        com posições $k,q\in\Phi=\{I,O\}\cap\Psi$.
    
        As bobinas $A_{in}\subseteq{A}$ podem ser transportadas de $I$ para o 
        armazém no intervalo de tempo contínuo $[\sigma^-, \sigma^+]$ e
        conjunto $A_{out}\subseteq{A}$ pode ser transportado para $O$ no intervalo
        $[\omega^-,\omega^+]$. Nisto também consideramos o tempo de movimentação
        $t^{load}_{kq}$ e a energia gasta $E^{load}_{kq}$ para movimentar a ponte
        rolante carregada para o espaço $(k,q)$. Similarmente, tempo de movimentação
        para uma movimentação descarregada $t^{empty}_{kq}$ e o consumo energético 
        $E^{empty}_{kq}$ também ja são definidos. 

        O tempo também é separado em seções $S$ que servem para limitar somente 
        um movimento da ponte rolate para cada seção $s,\hat{s}\in\{1...|S|\}$.

        As variáveis de decisão $\tau^s\in\mathbb{R}$ simbolizam o tempo de 
        início de cada seção $s$. Uma movimentação no estágio $s$ carregada com 
        a bobina $a$ para o espaço $(k,q)$ é definida por $W^{s}_{qk,a}$ e é uma 
        variável binária onde 1 indica a occorrencia da movimentação e 0 o 
        contrário. De forma parecida uma movimentação descarregada na seção $s$
        para a posição $(k,q)$ é denotada por $V^{s}_{k,q}$. Finalmente a variável 
        $x^{s}_{q,a}$ denota que a bobina $a$ está armazenada na posição $q$ 
        durante o intervalo $s$. 

        De acordo com \cite{YUAN2017424} As posições $k$ e $q$ são modeladas de 
        acordo com um array de três inteiros que delimitam as posições em largura,
        profundidade e altura e elas são definidas de acordo com a disposição de
        espaço do armazáem otimizado. 

        \subsection{Modelagem PLIM}

        O objetivo de otimização é a redução do custo energético de operação da 
        ponte rolante dado o espaço delimitado. Para tal queremos minimizar o 
        custo total de movimentação da ponte rolante carregada e vazia.

        \begin{align*}
            \min E &= \sum_{k \in \Phi} \sum_{q \in \Phi} \sum_{s \in S} \sum_{a \in A} 
                    \left( E_{kq,a}^{\text{load}} \cdot W_{kq,a}^s \right) \\
                &\quad + \sum_{k \in \Phi} \sum_{q \in \Phi} \sum_{s \in S} 
                    \left( E_{kq}^{\text{empty}} \cdot V_{kq}^s \right)
            \label{eq:objetivo}
        \end{align*}

        %% mais opções de formatação em https://claude.ai/public/artifacts/21eeb3b8-a6b8-4752-9433-9b1826e63fe4
        \subsection{Restrições}

        \begin{equation}
            \tau^{1} = 0
            \label{eq:t0}
        \end{equation}

        A igualdade \ref{eq:t0} implica que o tempo inicial modelado é o instante 0.

        \begin{equation}
            \sum_{q \in \Phi | q \neq I} \sum_{s \in S} W^s_{Iq,a} = 1 \quad \forall a \in A_{in} 
            \label{eq:in_movement_limit}
        \end{equation}

        \begin{equation}
            \sum_{k \in \Phi | k \neq O} \sum_{s \in S} W^s_{kO,a} = 1 \quad \forall a \in A_{out}
            \label{eq:out_movement_limit}
        \end{equation}

        A equação \ref{eq:in_movement_limit} 
        garantem que todas as cargas posicionadas na área de entrada são transportadas
        para o armazem ou para a saída diretamente. E  \ref{eq:out_movement_limit} 
        implica que todas as bobinas que devem ser entreques em $S$ são entregues.


        \begin{align}
            \sum_{k \in \Phi | k \neq O} \sum_{s \in S} W^s_{kO,a} = 0 \quad \forall a \in A \setminus A_{out} 
            \label{eq:storage_requirement}
        \end{align}

        Em contraste \ref{eq:storage_requirement} trata dos casos em que a bobina fica armazenada, portanto não
        entregue em $S$.
        
        

        \begin{align}
        \tau^s + \sigma^+_a \leq (1 - x^s_{I,a}) \cdot M \quad \forall a \in A_{in}, s \in S 
            \label{eq:storage_in_time}
        \end{align}
        
        \begin{align}
             \sigma^-_a - \tau^s \leq x^s_{I,a} \cdot M \quad \forall a \in A_{in}, s \in S 
            \label{eq:storage_deadline_time}
        \end{align}
        
        As equações \ref{eq:storage_in_time} e \ref{eq:storage_deadline_time} definem
        prioridades de movimentação para os momentos em que as cargas ficam disponíveis
        para armazenamento e o tempo limite de entrega respectivamente.

        \begin{align}
            \omega_a^- - \Bigg( \tau^s + 
            \sum_{k \in \Phi \mid k \neq O} \left( W^s_{kO,a} \cdot t^{load}_{kO} \right) 
            \Bigg) 
            &\leq (1 - x^s_{O,a}) \cdot M \notag \\
            &\forall a \in A_{out},\ s \in S 
            \label{eq:out_time_start}
        \end{align}
        A equação \ref{eq:out_time_start} evita que uma bobina seja entregue antes
        do tempo limite de entrega.

        \begin{align}
            s \geq\; & s^{-1} 
            + \sum_{k \in \Phi} \sum_{q \in \Phi} \left( t^{empty}_{kq} \cdot V^{s-1}_{kq} \right) \notag \\
            & + \sum_{k \in \Phi} \sum_{q \in \Phi} \sum_{a \in A} \left( t^{load}_{kq} \cdot W^{s-1}_{kq,a} \right) \notag \\
            & \forall\, s \in S,\ s \neq 1 \label{eq:time_progression}
        \end{align}

        A equação \ref{eq:time_progression} define a progressão do tempo em cada
         seção $s$.
        Note que cada intervalo não $s$ não sobrepoe outros intervaos e inclusíve
        essa formulação deixa aberta a possibilidade de tempo de espera entre fases.
        
        \begin{align}
            \sum_{k \in \Phi} x^s_{ka} = 1 \quad \forall a \in A,\ s \in S
            \label{eq:storage_only_one_coil}
        \end{align}

        \begin{align}
            \sum_{a \in A} x^s_{ka} \leq 1 \quad \forall a \in A, s \in S
            \label{eq:storage_only_one_section}
        \end{align}

        A equação \ref{eq:storage_only_one_coil} implica que cada bobina pode 
        ocupar apenas uma posição no espaço disponível. Isso em conjunto com 
        \ref{eq:storage_only_one_section} implica que cada espaço disponível 
        pode conter ao máximo uma bobina.

        \begin{align}
            \sum_{a \in A}
            \sum_{k \in \Phi}
            W^s_{kI, a} = 0 \quad \forall s \in S 
            \label{eq:storage_to_in_block}
        \end{align}

        \begin{align}
            \sum_{a \in A}
            \sum_{k \in \Phi}
            W^s_{Ok, a} = 0 \quad \forall s \in S 
            \label{eq:storage_to_out_block}
        \end{align}
        
        As equações \ref{eq:storage_to_in_block} e \ref{eq:storage_to_out_block}
        restringem movimentações de cargas para o espaço de entrada e a partir do
        espaço de saída.

        \begin{align}
            \sum_{k \in \Phi} \sum_{q \in \Phi} V^s_{kq} + \sum_{k \in \Phi} \sum_{q \in \Phi} \sum_{a \in A} W^s_{kq,a} \leq 1 \quad \notag \\
            \forall s \in S
            \label{eq:one_crane_movement}
        \end{align}

        A equação \ref{eq:one_crane_movement} implica que apenas uma movimentação
        pode ocorrer em cada seção $s$.

        \begin{align}
            \sum_{q \in \Phi} \sum_{a \in A} W^s_{kq,a} = \sum_{q \in \Phi} V^{s-1}_{qk} \quad \forall k \in \Phi, s \in S |_{s \neq 1}
            \label{eq:movement_precence_ld_after_empty}
        \end{align}

        Esta equação define a precedência entre movimentações de cargas. Ou seja,
        a ponte rolante inicia seu movimento carregada na seção $s$ apenas quando
        a seção $s-1$ vazia $V^{s-1}_{qk}$ terminou.

        \begin{align}
            \sum_{q \in \Phi} V^s_{kq} - \sum_{q \in \Phi} \sum_{a \in A} W^{s-1}_{qk,a} \leq 0 \quad \forall k \in \Phi, s \in S |_{s \neq 1}
            \label{eq:movement_precence_empty_after_ld}
        \end{align}

        Esta equação define a precedência entre movimentações vazias. Ou seja,
        um movimento vazio na seção $s$ ocorre apenas quando
        a seção $s-1$ carregada $W^{s-1}_{qk,a}$ terminou.

        \begin{align}
            x^s_{ka} = x^1_{ka} - \sum_{q \in \Phi} \sum_{\hat{s}=1}^{s-1} W^{\hat{s}}_{kq,a} + \sum_{q \in \Phi} \sum_{\hat{s}=1}^{s-1} W^{\hat{s}}_{qk,a} \quad \notag \\
            \forall s \in S |_{s \neq 1}, k \in \Phi, a \in A
            \label{eq:current_position_depends_on_previous_moves}
        \end{align}

        A equação \ref{eq:current_position_depends_on_previous_moves} define
        que a ocupação de um espaço vazio em um intervalo $s$ depende das movimentações
        de entrada e saída de bobinas em intervalos anteriores.


        \begin{align}
            \sum_{q \in \Phi} \sum_{a \in A} W^s_{kq,a} \leq 1 - \sum_{a \in A} x^s_{k+1,a} \quad \forall s \in S, k \in \Psi^1 \setminus \Psi^{P_{max}}
            \label{eq:upper_layer_blocking}
        \end{align}

        \ref{eq:upper_layer_blocking} garante que uma bobina no nível 1 fica 
        bloqueada para movimentação caso haja uma bobina no nível 2 em uma posição
        maior.

        \begin{align}
            \sum_{q \in \Phi} \sum_{a \in A} W^s_{kq,a} \leq 1 - \sum_{a \in A} x^s_{k-1,a} \quad 
            \forall s \in S, k \in \Psi^1 \setminus \Psi^{P_{1}}
            \label{eq:upper_layer_blocking_1}
        \end{align}

        Similarmente, \ref{eq:upper_layer_blocking_1} garante que uma bobina no nível 2 fica 
        bloqueada para movimentação caso haja uma bobina no nível 1 em uma posição
        menor.

        \begin{align}
            2 \cdot \sum_{k \in \Phi} \sum_{a \in A} W^s_{kq,a} \leq \sum_{a \in A} (x^s_{q-1,a} + x^s_{q+1,a}) \quad \notag \\
            \forall s \in S, q \in \Psi^2
            \label{eq:lower_layer_blocking}
        \end{align}

        Finalmente \ref{eq:lower_layer_blocking} garante que uma bobina no nível 2 fica 
        bloqueada de ser posicionada em um espaço caso não hajam duas bobinas 
        embaixo.

    \section{Geração de dados para testes}

        A implementação da solução depende da obtenção ou geração de dados que
        representam a produção, espaço de armazenagem, custos de movimentação, 
        tempo de armazenagem e a demanda por bobinas de aço.
        Devido a imensa complexidade da coleta e riscos de vazamentos de dados 
        industriais, nossa equipe optou por não usar dados reais.
        %% Preciso de rever o parágrafo abaixo
        A geração de dados foi feita com base em equações simples de mecânica,
        regras de agendamento para entrada e saída de cargas do armazém. Nisto também
        foram delimitados parâmetros com números pequenos para que a solução suporte
        a possibilidade de escala por meio do princípio da linearidade.
        
        A fim de testar a viabilidade da implementação mais rápido supomos que 
        todas as bobinas têm o mesmo peso e que a ponte rolante faz moviventos 
        horizontais na diagonal e movimentos verticais somente no eixo y. Dessa
        foma os custos de tempo e energia podem ser calculados por:

        \begin{align}
            E^{load}_{kqa} = w_a d^{vertical}_{kq} + w_a d^{horizontal}_{kq}
        \end{align}

        \begin{align}
            E^{empty}_{kq} = d^{vertical}_{kq} + d^{horizontal}_{kq}
        \end{align}

        Similarmente os tempos de movimentação de das bobinas ente posições é 
        calculado de forma proporcional a distancia entre espaços. 

        \begin{align}
            t^{load}_{kqa} = (d^{vertical}_{kq} + d^{horizontal}_{kq}) \Theta
        \end{align}

        \begin{align}
            t^{empty}_{kq} = (d^{vertical}_{kq} + d^{horizontal}_{kq}) \Theta
        \end{align}

        Onde $\theta$ é uma constante arbitrária que, em uma situação mais realista,
        dependeria da velocidade máxima da ponte rolante.

    \subsection{Implementação do Algoritmo}
    
    O algoritmo foi implementado utilizando a biblioteca Gurobi para otimização.
    Para isto foram feitas estruturas em python para que que geram os paramêtros 
    de entrada e os conjuntos usados nas restrições do problema. A função $gerar_modelo$
    conteve todas as 18 restrições determinadas pela modelagem.
    
    \subsection{Resultados}

    As solução deste problema pelo solver MIP não é trivial, portanto
    foram feitas diversas iterações de testes para garantir que o modelo
    se comportasse de forma esperada. A seguir estão os principais pontos
    de atenção que foram observados durante a implementação:
    \begin{itemize}
        \item Com a atual modelagem é possível que o solver não encontre uma solução
        \item A modelagem não funcionou para casos em que hávia bobinas entrando e saíndo.
        \item Mesmo que não haja bobinas entrando e saindo, a modelagem pode não encontrar uma solução viável.
        \item Caso o número de intervalos $s$ seja 1 o modelo geralmente converge porque há somente uma movimentação descarregada.
    
    \section{Conclusão}            

    Este trabalho explorou a otimização do movimento de bobinas de aço com 
    o objetivo de reduzir o consumo energético dos guindastes. Embora a relevância
    do tema seja inegável, dada a importância do aço na economia global e o
    impacto ambiental do gasto de energia na movimentação, a complexidade inerente
    ao problema impôs desafios significativos.

    A vasta quantidade de restrições operacionais, dificultaram a execução de testes
    para provar a pertinência da modelagem do problema. Consequentemente, a geração 
    de resultados robustos, para a otimização do movimento das bobinas e a consequente 
    redução do consumo energético, não foi alcançada dentro do escopo deste estudo.

    Ainda que os objetivos iniciais não tenham sido plenamente atingidos, em termos de 
    resultados quantitativos, este estudo serviu para evidenciar a complexidade da otimização
    logística em ambientes industriais e a necessidade de abordagens mais sofisticadas.
    Para futuros trabalhos, vislumbra-se a possibilidade de decompor o problema em
    subproblemas menores, além da aplicação de heurísticas ou meta-heurísticas.
    
    \bibliography{referencias}

\end{document}
